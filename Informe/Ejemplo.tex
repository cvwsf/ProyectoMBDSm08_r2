\documentclass{article}
\usepackage{multicol}
\usepackage{booktabs} 
\usepackage{blindtext}

\renewcommand{\refname}{Referencias}
\renewcommand{\tablename}{Tabla}

\title{Ejemplo de Documento Sweave}
\author{Carlos Schmidt}
\date{2024-07-22}

\usepackage{Sweave}
\begin{document}
\Sconcordance{concordance:Ejemplo.tex:Ejemplo.Rnw:1 7 1 1 0 11 1 1 3 2 0 1 1 7 0 1 2 %
9 1 1 6 5 0 1 21 19 0 1 3 4 0 1 2}



\begin{Schunk}
\begin{Sinput}
> # Instalar y cargar el paquete tinytex si no está cargado
> if (!requireNamespace("tinytex", quietly = TRUE)) {
+   install.packages("tinytex")
+ }
> library(tinytex)
\end{Sinput}
\end{Schunk}


\maketitle

\section*{Primeros pasos}

\begin{enumerate}
\item Punto 1
\item Punto 2
\item Punto 3

\end{enumerate}




Ejemplo de código de R ejecutable en fragmento de texto
La base de datos \texttt{df_filtered} tiene \emph{25827} observaciones.


\begin{multicols}{2}
[
\section{Texto con formato}
Algún texto aleatorio
]
\noindent \underline{Cursiva}\\  \emph{\blindtext}   \\   \underline{Typewritter}  \\   \texttt{\blindtext}
\end{multicols}


\begin{multicols}{2}
[
\subsection{Tamaño de fuente}
Modificando tamaño de fuente con palabras reservadas en \LaTex
]
{\Huge \blindtext}   {\scriptsize \blindtext}
\end{multicols}


\section{Tablas en LaTex}
\subsection{Tablas generadas desde Latex}
\begin{table}[h!]
\caption{Esto es un ejemplo de una tabla simple en Latex.}
\begin{tabular}{1 | c c}
\hline
\bf{Observación} & \bf{Var1} & \bf{Var2} \\
\hline
Individuo 1 & a & b \\
Individuo 2 & c & d \\
Individuo 3 & e & f \\
Individuo 4 & g & h \\
\hline
\end{tabular}
\end{table}



\subsection{Tablas generadas desde R}
\begin{table}[h!]
\centering
\caption{Resumen del Modelo}
\begin{tabular}{@{}lrrrr@{}}
\toprule
Variable & Estimate & Std. Error & t value & Pr(>|t|) \\
\midrule
(Intercept) &  1.6571639 & 0.25594569 &  6.474670 & 1.400102e-09 \\
Sepal.Length &  0.3777734 & 0.06556897 &  5.761466 & 4.867516e-08 \\
Petal.Length & -0.1875666 & 0.08349289 & -2.246498 & 2.619373e-02 \\
Petal.Width &  0.6257105 & 0.12337631 &  5.071561 & 1.195649e-06 \\
Speciesversicolor & -1.1602853 & 0.19329450 & -6.002681 & 1.503304e-08 \\
Speciesvirginica & -1.3982549 & 0.27714618 & -5.045189 & 1.344532e-06 \\
\bottomrule
\end{tabular}
\end{table}
begin{table}[h!]
centering
caption{Resumen del Modelo}
begin{tabular}{@{}lrrrr@{}}
toprule
Variable & Estimate & Std. Error & t value & Pr(>|t|) \
midrule
(Intercept) &  1.6571639 & 0.25594569 &  6.474670 & 1.400102e-09 \
Sepal.Length &  0.3777734 & 0.06556897 &  5.761466 & 4.867516e-08 \
Petal.Length & -0.1875666 & 0.08349289 & -2.246498 & 2.619373e-02 \
Petal.Width &  0.6257105 & 0.12337631 &  5.071561 & 1.195649e-06 \
Speciesversicolor & -1.1602853 & 0.19329450 & -6.002681 & 1.503304e-08 \
Speciesvirginica & -1.3982549 & 0.27714618 & -5.045189 & 1.344532e-06 \
bottomrule
end{tabular}
end{table}




\begin{Schunk}
\begin{Sinput}
> # Para establecer globalmente tamaño de gráficos
> knitr::opts_chunk$set(echo = FALSE, fig.height = 10, fig.width = 10)
\end{Sinput}
\end{Schunk}


Ejemplo de código de R ejecutable en fragmento de texto
La base de datos \texttt{iris} tiene \emph{150} observaciones.



\end{document}

\begin{Schunk}
\begin{Sinput}
> 
\end{Sinput}
\end{Schunk}

