\documentclass{article}
\usepackage{multicol}

\renewcommand{\refname}{Referencias}
\renewcommand{\table}{Tabla}

\title{Ejemplo de Documento Sweave}
\author{Carlos Schmidt}
\date{2024-07-21}






\usepackage{Sweave}
\begin{document}
\Sconcordance{concordance:Ejemplo.tex:Ejemplo.Rnw:1 12 1 1 0 15 1 1 8 1 5 6 1 1 30 19 %
1 1 35 1 2 6 1 1 27 18 1 1 36 1 2 6 1 1 27 17 1 1 45 1 2 38 1}


\maketitle

\section*{Primeros pasos}

\begin{enumerate}
\item Punto 1
\item Punto 2
\item Punto 3

\end{enumerate}


\begin{Schunk}
\begin{Sinput}
> if(!require(tinytex)){
+   install.packages('tinytex')
+   library(tinytex)
+   
+ }
\end{Sinput}
\end{Schunk}


\begin{Schunk}
\begin{Sinput}
> # Para instalar otros paquetes
> tinytex::tlmgr_install('blindtext')
\end{Sinput}
\end{Schunk}

\begin{multicols}{2}
[
\section{Texto con formato}
Algún texto aleatorio
]
\noindent \underline{Cursiva}\\  \emph{\blindtext}   \\   \underline{Typewritter}  \\   \texttt{\blindtext}
\end{multicols}


\begin{multicols}{2}
[
\subsection{Tamaño de fuente}
Modificando tamaño de fuente con palabras reservadas en \LaTex
]
{\Huge \blindtext}   {\scriptsize \blindtext}
\end{multicols}


\section{Tablas en LaTex}
\subsection{Tablas generadas desde Latex}
\begin{table}[h!]
\caption{Esto es un ejemplo de una tabla simple en Latex.}
\begin{tabular}{1 | c c}
\hline
\bf{Observación} & \bf{Var1} & \bf{Var2} \\
\hline
Individuo 1 & a & b \\
Individuo 2 & c & d \\
Individuo 3 & e & f \\
Individuo 4 & g & h \\
\hline
\end{tabular}
\end{table}



\subsection{Tablas generadas desde R}
\begin{Schunk}
\begin{Sinput}
> # El encabezamiento de arriba sirve para tener la tabla sin el código que el sistema genera para generarla
> data(iris)
> mod <- lm(Sepal.Width ~ ., data=iris)
> xtable::xtable(summary(mod))
\end{Sinput}
% latex table generated in R 4.3.2 by xtable 1.8-4 package
% Sun Jul 21 20:23:57 2024
\begin{table}[ht]
\centering
\begin{tabular}{rrrrr}
  \hline
 & Estimate & Std. Error & t value & Pr($>$$|$t$|$) \\ 
  \hline
(Intercept) & 1.6572 & 0.2559 & 6.47 & 0.0000 \\ 
  Sepal.Length & 0.3778 & 0.0656 & 5.76 & 0.0000 \\ 
  Petal.Length & -0.1876 & 0.0835 & -2.25 & 0.0262 \\ 
  Petal.Width & 0.6257 & 0.1234 & 5.07 & 0.0000 \\ 
  Speciesversicolor & -1.1603 & 0.1933 & -6.00 & 0.0000 \\ 
  Speciesvirginica & -1.3983 & 0.2771 & -5.05 & 0.0000 \\ 
   \hline
\end{tabular}
\end{table}\end{Schunk}


\begin{Schunk}
\begin{Sinput}
> # Para establecer globalmente tamaño de gráficos
> knitr::opts_chunk$set(echo = FALSE, fig.height = 10, fig.width = 10)
\end{Sinput}
\end{Schunk}


Ejemplo de código de R ejecutable en fragmento de texto
La base de datos \texttt{iris} tiene \emph{150} observaciones.



\end{document}

\begin{Schunk}
\begin{Sinput}
> 
\end{Sinput}
\end{Schunk}

